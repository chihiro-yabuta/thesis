\section{優美動作の評価}

\subsection{従来手法}
上田研ではこれまで多くのモデルが提唱されてきた.
その多くは稲津\cite{inazu2}の全曲率計算を起源としている.
先に挙げたB-spline近似で手先軌道を曲線近似し,図\ref{curves}のように
全曲率$\mu$が0.87〜1.31となる曲線を多く含むものを優美としている.
そこから面積や角度に派生するモデルも存在するが,根底は手先軌道である.
今回作成したネットワークが「優美」と判定したものは動画のどの箇所を根拠に
「優美」と判断したのか,動作を抽出して検証する際,どこに着目すべきと
示唆しているかなどを検証した.

\begin{figure}[b]
  \begin{center}
    \includegraphics[width=100mm]{images/quote/curves.png}
  \end{center}
  \caption{Hogarth Curveの全曲率}
  \label{curves}
\end{figure}

\subsection{Grad Camを用いた評価}
作成したネットワークの判断根拠を可視化するために,Pytorch-GradCam\cite{pygradcam}を
使用した.作成したネットワークはPytorch\cite{pytorch}でできているので,GradCamも
同じくPytorchでできたものを使用した.

Pytorch-GradCamは入力データに対してどのように各層が影響するか順伝播で確認してから,
逆伝播で影響の強弱を1以下の数値で返す.今回のネットワークでは動画をフレーム毎に区切るので
1フレームずつ処理すると図\ref{camgraph}のように最大30回重複するデータが存在する.
よって出力された結果を順番に足し合わせ,特定のデータについて足し合わされた回数で各値を割ることで
データの均一化を図った.データは1以下にあるので255をかけることでデータの強弱を動画として可視化した.
また,数値は小さいが,データが離散しているので閾値0.4以下の値は0とした.

\begin{figure}[b]
  \begin{center}
    \includegraphics[width=120mm]{images/chart/gradcam.pdf}
  \end{center}
  \caption{Grad Camから出力される結果の整形方法}
  \label{camgraph}
\end{figure}
\clearpage

\begin{table}[t]
  \begin{center}
    \begin{tabular}{|c|cc|} \hline
      優美なダンス
        & \includegraphics[width=50mm]{images/cam/chinese.png}
        & \includegraphics[width=50mm]{images/cam/japanese.png}
      \\ \hline
      普通のダンス
        & \includegraphics[width=50mm]{images/cam/kadokawa.png}
        & \includegraphics[width=50mm]{images/cam/aito.png}
      \\ \hline
      その他の動作
        & \includegraphics[width=50mm]{images/cam/radio.png}
        & \includegraphics[width=50mm]{images/cam/running.png}
      \\ \hline
    \end{tabular}
  \end{center}
  \caption{観測された動画特徴例}
  \label{examples}
\end{table}

このようにして出力されたGrad Camの動画と二値化動画を横に繋ぎ合わせ,どのような特徴があるか
観察した.すると表\ref{examples}のような特徴を観測した.
ここで判断根拠分布は
\begin{enumerate}
  \item 優美なダンスは画面「下部」に「広く」分布している \\
        領域を大きく使っていることから手足を頻繁に使う
  \item 普通のダンスは画面「中部」に「小さく」分布している \\
        領域を狭く使っていることから体幹移動を頻繁に使う
  \item その他の動作は画面「上部」に「広く」分布している \\
        判断根拠が離散している
\end{enumerate}
という仮説を立てた.この仮説を検証するため,それぞれの動画に対して
縦に三つに分割した時の判断根拠分布率を算出した.
\clearpage

\begin{figure}[t]
  \begin{center}
    \includegraphics[width=120mm]{images/chart/divide.pdf}
  \end{center}
  \caption{Grad Cam動画の画素計算方法}
  \label{divide_graph}
\end{figure}

図\ref{divide_graph}のようにGrad Cam動画データを縦に三分割した.
黒画素以外の個数を数え,それを指定領域数で割ることでその領域に占める
判断根拠の分布率を算出した.

その結果,表\ref{devide_summary}のような数値を得た.いくつかの動画について,
優美なダンスでは「中部」「下部」に広く分布していることが分かった.
いくつかの動画では,領域を大きく使っていることから
手足を頻繁に使う動作が優美と判定されているように思う.
逆に,普通のダンスはいくつかの動画では分布が狭い,つまり画面中央に密集しており,
体幹移動を頻繁に使うのではないかと考える.
しかし,入力データに複数人で踊っているデータが存在していたり,精度の悪い動画や
先の仮説にそぐわない動画も存在しているので相関は薄く推測の域は出ない.

\begin{table}[t]
  \begin{center}
    \begin{tabular}{|c|p{5mm}p{5mm}p{5mm}|p{5mm}p{5mm}p{5mm}|p{5mm}p{5mm}p{5mm}|p{5mm}p{5mm}p{5mm}|} \hline
        & \multicolumn{6}{|c|}{学習用} & \multicolumn{6}{|c|}{推論用} \\ \cline{2-13}
        &上 &中 &下 &上 &中 &下 &上 &中 &下 &上 &中 &下 \\ \hline
        &2.1 &34.4 &62.8 &10.8 &16.9 &48.8 &0.0 &43.2 &38.2 &40.6 &71.2 &72.4 \\
        & \multicolumn{3}{|c|}{\includegraphics[width=18mm]{images/snaps/japanese_elegant.png}}
        & \multicolumn{3}{|c|}{\includegraphics[width=18mm]{images/snaps/chinese_elegant.png}}
        & \multicolumn{3}{|c|}{\includegraphics[width=18mm]{images/snaps/ballet_group_elegant.png}}
        & \multicolumn{3}{|c|}{\includegraphics[width=18mm]{images/snaps/japanese_group_elegant.png}}
      \\ \cline{2-13}
      優美
        &6.8 &60.0 &31.4 &48.3 &71.6 &33.8 &1.3 &48.1 &54.7 &6.3 &28.5 &52.1 \\
        & \multicolumn{3}{|c|}{\includegraphics[width=18mm]{images/snaps/ballet_elegant.png}}
        & \multicolumn{3}{|c|}{\includegraphics[width=18mm]{images/snaps/thai_elegant.png}}
        & \multicolumn{3}{|c|}{\includegraphics[width=18mm]{images/snaps/chinese_group_elegant.png}}
        & \multicolumn{3}{|c|}{\includegraphics[width=18mm]{images/snaps/belly_elegant.png}}
      \\ \cline{2-13}
        &28.0 &11.7 &24.7 & & & & & & & & & \\
        & \multicolumn{3}{|c|}{\includegraphics[width=18mm]{images/snaps/japanese2_elegant.png}}
        & \multicolumn{3}{|c|}{}
        & \multicolumn{3}{|c|}{}
        & \multicolumn{3}{|c|}{}
      \\ \hline
        &16.0 &42.5 &36.6 &21.4 &32.0 &27.4 &79.8 &72.9 &59.5 &13.8 &75.5 &22.4 \\
        & \multicolumn{3}{|c|}{\includegraphics[width=18mm]{images/snaps/ariana_dance.png}}
        & \multicolumn{3}{|c|}{\includegraphics[width=18mm]{images/snaps/kadokawa_dream_dance.png}}
        & \multicolumn{3}{|c|}{\includegraphics[width=18mm]{images/snaps/bts_group_dance.png}}
        & \multicolumn{3}{|c|}{\includegraphics[width=18mm]{images/snaps/arashi_group_dance.png}}
      \\ \cline{2-13}
      普通
        &11.4 &38.4 &13.6 &31.6 &47.9 &50.3 &12.5 &26.2 &4.8 &15.4 &8.3 &22.4 \\
        & \multicolumn{3}{|c|}{\includegraphics[width=18mm]{images/snaps/bts_dance.png}}
        & \multicolumn{3}{|c|}{\includegraphics[width=18mm]{images/snaps/manolo_dance.png}}
        & \multicolumn{3}{|c|}{\includegraphics[width=18mm]{images/snaps/hyoga_dance.png}}
        & \multicolumn{3}{|c|}{\includegraphics[width=18mm]{images/snaps/legit_dance.png}}
      \\ \cline{2-13}
        &19.6 &63.8 &54.8 & & & & & & & & & \\
        & \multicolumn{3}{|c|}{\includegraphics[width=18mm]{images/snaps/aito_dance.png}}
        & \multicolumn{3}{|c|}{}
        & \multicolumn{3}{|c|}{}
        & \multicolumn{3}{|c|}{}
      \\ \hline
        &52.6 &59.3 &21.2 &32.0 &33.3 &31.3 &41.8 &57.8 &22.3 & & & \\
        & \multicolumn{3}{|c|}{\includegraphics[width=18mm]{images/snaps/radio_exer.png}}
        & \multicolumn{3}{|c|}{\includegraphics[width=18mm]{images/snaps/posing.png}}
        & \multicolumn{3}{|c|}{\includegraphics[width=18mm]{images/snaps/radio_exer_2.png}}
        & \multicolumn{3}{|c|}{}
      \\ \cline{2-13}
      他
        &35.4 &68.7 &78.3 &15.5 &62.3 &75.9 & & & & & & \\
        & \multicolumn{3}{|c|}{\includegraphics[width=18mm]{images/snaps/shadowboxing.png}}
        & \multicolumn{3}{|c|}{\includegraphics[width=18mm]{images/snaps/running.png}}
        & \multicolumn{3}{|c|}{}
        & \multicolumn{3}{|c|}{}
      \\ \cline{2-13}
        &81.5 &49.5 &45.2 &40.0 &52.5 &47.2 & & & & & & \\
        & \multicolumn{3}{|c|}{\includegraphics[width=18mm]{images/snaps/shinkokyu.png}}
        & \multicolumn{3}{|c|}{\includegraphics[width=18mm]{images/snaps/leaves.png}}
        & \multicolumn{3}{|c|}{}
        & \multicolumn{3}{|c|}{}
      \\ \hline
    \end{tabular}
  \end{center}
  \caption{算出された判断根拠分布率[%]}
  \label{devide_summary}
\end{table}
\clearpage

\subsection{確率分布を用いた評価}
別のアプローチから判断根拠を知るために,動画の確率分布を検証した.
これは動画が時間に依存したデータであることから着想を得た.
Grad Camの動画生成と同様の手法で,
図\ref{distchart}のように動画を1フレームずつ処理する.
確率分布の増減から,Grad Camより細かな判断根拠の特定と,
どのように動作するかを特定することを目指した.

ほとんどが表\ref{net_dist}のようにほとんど変化のないグラフとなったが,
精度の低いタイ舞踊では,動画の700〜800フレームに盛り上がりが見られた.
このフレームを確認すると動画のように手を大きく広げて舞踊しており,全体を通しても
優美だと感じた.クラス分類のネットワークでは出力が確率であるため
特徴を捉えることは難しかった.これは三値分類の確率であるため,
確率がほとんど1から変化しないためである.

\begin{figure}[b]
  \begin{center}
    \includegraphics[width=120mm]{images/chart/net_dist.pdf}
  \end{center}
  \caption{動画長確率分布の計算方法}
  \label{distchart}
\end{figure}

\begin{table}[b]
  \begin{center}
    \begin{tabular}{|c|} \hline
      \includegraphics[width=100mm]{images/dist/thai_elegant_900.pdf} \\ タイ:500〜900 \\ \hline
      \includegraphics[width=100mm]{images/dist/japanese_elegant_2000.pdf} \\ 日本:1500〜2000 \\ \hline
      \includegraphics[width=100mm]{images/dist/chinese_elegant_1500.pdf} \\ 中国:1000〜1500 \\ \hline
    \end{tabular}
  \end{center}
  \caption{確率分布出力結果のグラフと対象舞踊フレーム数}
  \label{net_dist}
\end{table}
\clearpage

\subsection{従来手法との比較}
今回の評価では大きな成果は得られなかったが,AIを用いることで
モデルベースでは時間的,フォーマット的に扱えなかったデータを扱うことができた.
しかし,特徴を漠然としか捉えることができず,現段階では,モデルベースと学習ベースの
二つの併用運用が考えられる.

表\ref{json}のように動画から体の部位を抽出できるAIフレームワークのOpenposeを使用して手先軌道を
取得し,特徴の意図を汲み取る試みも行ったが,外れ値がデータのほとんどを占めており,
youtubeから取得した画素数の少ない動画では,手先軌道の抽出は不可能であった.
Grad Camの結果から考察した仮説である
\begin{enumerate}
  \item 優美なダンスは手足を頻繁に使う
  \item 普通のダンスは体幹移動を頻繁に使う
  \item その他の動作は確率が分散している
\end{enumerate}
が真とすると,従来研究の手先に注目する解析は
今回のモデルとも共通項があるが,足運びに注目することを提案することも可能となる.

確率分布の可視化精度の向上として挙げられることとして,分類数を増やすことや,
同じ確率の数値であっても,特徴を取り出すような手法を
開発できればその確率の特徴を見ることが可能だと考える.

\begin{table}[b]
  \begin{center}
    \begin{tabular}{|c|} \hline
      \includegraphics[width=130mm]{images/dist/thai_elegant_json.pdf} \\ \hline
    \end{tabular}
  \end{center}
  \caption{Openposeから抽出したタイ舞踊:700〜800フレームの手先起動}
  \label{json}
\end{table}
\clearpage