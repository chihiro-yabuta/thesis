所属するヒューマンモデリング研究室では
長年「優美さ」の定量化を目標に様々なアプローチが試されてきた.
これまでのアプローチは人間の主観評価をベースに,手先軌道に着目した解析であった.
ここで,深層学習を用いて網羅的に優美さを特定しようとした場合,どのような結果をもたらすのか,疑問に思った.

本研究の目的は,この疑問を解消すべく,「優美さ」の根底にある判断根拠とは一体どんなものなのか,
従来の結果と深層学習の間にどのような相違点及び共通点が存在するのかを特定することである.

深層学習から直接に入力データのどこが優美か特定させることは困難だと考え,
まず動画をクラス分類することを考えた.
その後,学習から生じる各層の行列や出力される確率から
判断根拠を可視化することを目指した.

クラス分類として,入力された動画を
[優美なダンス,普通のダンス,その他の動作]に分類するネットワークを作成した.
ネットワーク内での分類手順として
\begin{enumerate}
  \item 動画に二値化処理を施す
  \item 作成したネットワーク処理で計算する
  \item 配列長3の二値データを得る
\end{enumerate}
のような順序で分類した.推論データに対して75%の精度で分類することができた.

次に,作成したネットワークの判断根拠を可視化した.
用いた手法として,Grad Camと確率分布を使用した.
Grad Camからは3つのカテゴリごとに判断根拠分布に偏りが見られた.
解析の結果,いくつかの動画で優美なダンスに手先や足運びに判断根拠が偏っていた.
また,確率分布では,手先を大きく広げる動作において,確率の高まりが見られた.

今回の評価では大きな成果は得られなかったが,
深層学習を用いることでモデルベースでは時間的,フォーマット的に扱えなかったデータを扱うことができた.
従来研究の手先に注目する解析は今回のモデルとも共通項があるが,足運びに注目することも必要であることが示唆された.