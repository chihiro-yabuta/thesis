\section{おわりに}
今回の研究では,深層学習から優美さを抽出することができた.

作成したネットワークは,現在開発の盛んである動画解析を
目的に作成されたものであるが,畳み込みとTransformer Encoderを併用して
使うネットワークであるConvolution Vision Transformer\cite{cvt}と
共通項が多々ある.

CvTは画像の学習に使われるが,
今回のネットワークから,バッチ対象をフレームにする処理を行うことで
動画にも応用できることが分かった.
畳み込み層がバッチ数だけ増加するため,計算時間とデータが増加してしまうが,
新たなフレームワークとして提案したい.

動画編集処理は更なる研究が必要である.
Grad Camの判断根拠分布が漠然としている原因は動画編集にあり,
特に修正の必要な箇所として,人間の輝度が黒で表現されているものと,
白で表現されているものがある.
目的とするデータの均一化にはそぐわず,人間の輝度を白で統一する必要がある.

画像から動画中央に位置する対象物を切り抜くことができる,
AIフレームワークのRembg\cite{Rembg}は,
画像一枚に対しても数秒の計算時間を要し,
動画の全フレームを処理するためには多大な時間を要した.

今回の研究は,ヒューマンモデリング研究室の新たな研究テーマの提案,
ひいては手法改善を目的としているので,
計算時間を要するRembgは使用しなかった.

しかし,輪郭の取得精度では今回の動画編集処理を大きく上回っており,
Rembgを使用して学習を進めれば,より鮮明に判断根拠分布を観測できる.
こちらも合わせて更なる研究の余地として提案したい.

深層学習の判断根拠可視化は学会でも議論の分かれるところでもあるが,今回の研究が舞踊解析の一助になれば幸いである.